\subsection*{Trabalho realizado por\+:}

\begin{DoxyVerb}* Guilherme Saturno a21700118
* Lucas Viana a21805095
* Rita Saraiva a21807278
\end{DoxyVerb}


\subsection*{Descrição da solução\+:}

\begin{DoxyVerb}* Utilizamos um "switch" dentro de um ciclo "while" para selecionar funções armazenadas num ficheiro Header(funtions.h), que utilizam um ciclo "for", estrutura "if, else" ou "scanf e printf" para recolher e devolver dados ao usuário.
* Utilizamos matrizes para armazenar os dados dos jogadores, tal como recursos, posições e pontos.
* Também utilizamos as matrizes para definir o mapa de jogo e as repectivas características de cada área.

![Fluxograma](/images/fluxo.jpg)
\end{DoxyVerb}


\subsection*{Manual de utilizador\+:}

\begin{DoxyVerb}Como compilar: 
    * make all(compilar tudo a partir do gcc);
    * make clean( para limpar os ficheiros .o e .out);
    * ./prog para iniciar o programa(executável já disponível).

Como jogar:
    O jogo trata-se de um player vs player jogado através de turnos. Começa com o player 1 a escolher uma aldeia e de seguida o player 2 a fazer a sua escolha. Depois, é apresentado uma lista de opções do player 1 que pode optar por:

    1 - Lançar os dados
    2 - Contruir uma aldeia
    3 - Evoluir uma aldeia para uma cidade
    4 - Trocar 10 materiais por 1 ponto
    5 - Trocar 4 materiais por uma carta à escolha
    6 - Ver Stats
    7 - Ver mapa
    8 - Comandos
    9 - Regras
    10 - Sair

    Cada vez que é lançado os dados o player altera possibilitando cada um ter todas as escolhas e ambos recebem recursos se a soma dos dados for um número que esteja presente nos terrenos adjacentes de sua aldeia. Se o player não tiver recursos ou a opção que escolheu não for possível realizar aparece uma mensagem e pede de novo qual a opção que quer optar por. O jogo termina quando um dos players chegarem a 6 pontos.
\end{DoxyVerb}


\subsection*{Conclusões e Matéria aprendida\+:}

\begin{DoxyVerb}Com este projeto, desenvolvemos a nossa aprendizagem de realizar ciclos "for" para manipular matrizes, como também utilizar um ficheiro Header e passar parâmetros para as funções contidas no mesmo.
Para além disso, ganhamos mais conhecimento referente ao Makefile, .gitignore e outros. Devido à falta de conhecimento perante à utilização do ficheiro .INI, criamos matrizes utilizando arrays de arrays para estabelecer as posições e os respetivos recursos(contendo o número calhado da soma dos dados) do mapa.
\end{DoxyVerb}


\subsection*{Referências\+:}

\begin{DoxyVerb}Retiramos dúvidas e ideias em:
    * Questões colocadas no site stackoverflow.com;
    * PDF feitas pelo Professor;
    * Vídeos com tutoriais;
    * Colegas do curso.\end{DoxyVerb}
 